\newglossaryentry{circular array}
{
    name={circular array},
    description={
        An array in which the element indexed directly after the last element 
        is effectively the first element, giving the effect of the 
        data-structure ``wrapping around".
    },
    plural={circular arrays}
}

\newglossaryentry{thread}
{
    name={thread},
    description={
        A strand of execution which operates independently from the main flow of
        control in a process.
    },
    plural={threads}
}

\newglossaryentry{multi-threaded algorithm}
{
    name={multi-threaded algorithm},
    description={
        A program which utilises more than one thread to achieve its desired 
        result.
    },
    plural={multi-threaded algorithms}
}

\newglossaryentry{scheduling}
{
    name={scheduling},
    description={
        The process of determining the order in which parts of a program access
        system resources.
    }
}

\newglossaryentry{dead-lock}
{
    name={dead-lock},
    description={
        A situation in which two or more threads are waiting for each other to 
        give up exclusive access to a resource required to continue execution.
    }
}

\newglossaryentry{worker-thread}
{
    name={worker-thread},
    description={
        A thread which, in general, performs tasks that work towards achieving 
        the purpose of the program.
    },
    plural={worker-threads}
}

\newglossaryentry{monitor-thread}
{
    name={monitor-thread},
    description={
        A thread which performs administrative tasks such as maintaining worker-threads
        and controlling synchronisation.},
    plural={monitor-threads}
}

\newglossaryentry{mutex}
{
    name={mutex},
    description={
        In the context of pthreads, a mutex is a mechanism
        for acquiring exclusive lock on a resource. Short for mutual exclusion.
    },
    plural={mutexes}
}

\newglossaryentry{barrier}
{
    name={barrier},
    description={
        A thread synchronisation technique in which a specified number of threads 
        must reach an explicit point in execution before any may continue.    
    },
    plural={barriers}
}

\newglossaryentry{condition-variable}
{
    name={condition-variable},
    description={
        A thread synchronisation technique which allows conditional lock of a resource dependant
        on the state of some value.
    },
    plural={condition-variables}
}
\newglossaryentry{load-balancing}
{
    name={load-balancing},
    description={
        A method in which work-load is distributed across multiple resources in an 
        attempt to optimise utilisation of such resources.
    }
}

\newglossaryentry{work-stealing}
{
    name={work-stealing},
    description={A processor which is starved of work attempts to ``steal'' 
                 work from other processors.}
}

\newglossaryentry{thief}
{
    name={thief},
    description={
        A thread which has the work-load of some other thread re-assigned to its own.
    },
    plural={thieves}
}

\newglossaryentry{victim}
{
    name={victim},
    description={
        A thread which has a portion of its work-load re-assigned to some other thread.
    },
    plural={victims}
}

\newglossaryentry{ready-deque}
{
    name={ready-deque},
    description={
        A queue data-structure which allows access to both the top and bottom elements.
    },
    plural={ready-deques}
}

\newglossaryentry{non-blocking}
{
    name={non-blocking},
    description={
        A property of a multi-threaded algorithm in which shared resources are not
        locked, allowing threads to operate without being stopped.
    }
}

\newglossaryentry{steal-operation}
{
    name={steal-operation},
    description={
        The process of re-distributing work-load from a victim to a thief.
    }
}

\newglossaryentry{work-sharing}
{
    name={work-sharing},
    description={
        A processor which creates new work attempts to migrate it to 
        another underutilised processor at creation time.
    }
}

\newglossaryentry{locality}
{
    name={locality},
    description={
        The amount to which a value, or set of related values, is utilised by a resource.
    }
}

\newglossaryentry{critical-orbit}
{
    name={critical-orbit},
    plural={critical-orbits},
    description={
        The orbit of point \(0\) for a set of complex numbers.
    }
}

\newglossaryentry{fractal}
{
    name={fractal},
    plural={fractals},
    description={
        A set which has a fractional dimension.
    }
}

\newglossaryentry{complex-number}
{
    name={complex-numer},
    plural={complex-numbers},
    description={
        A number which is expressed in two parts; real and imaginary \((a + b_i)\).
    }
}

\newglossaryentry{real-world-fractal}
{
    name={real-world-fractal},
    description={
        Fractals which do not exhibit the property of scale invariance. 
    },
    plural={real-world-fractals}
}

\newglossaryentry{mathematical-fractal}
{
    name={mathematical-fractal},
    description={
        Fractals which could exhibit the property of infinite scale invariance or
        quasi-self similarity.
    },
    plural={mathematical-fractals}
}

\newglossaryentry{scale-invariance}
{
    name={scale-invariance},
    description={
        A property of a shape which is identical at magnifications of a common factor.
    }
}

\newglossaryentry{self-similarity}
{
    name={self-similarity},
    description={
        A property of a shape which is comprised of part of itself.
    }
}

\newglossaryentry{exact self-similarity}
{
    name={exact self-similarity},
    description={
        A property of a shape which is comprised of an exact copy of part of itself.
    }
}

\newglossaryentry{quasi self-similarity}
{
    name={quasi self-similarity},
    description={
        A property of a shape which is comprised of an approximate copy of part of itself.
    },
    plural={quasi self-similar}
}
