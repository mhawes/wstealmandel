% ------------------------------------------------------------------------------
% includes preamble stuff
\documentclass[a4paper,11pt]{report}
%\documentclass[a4paper,11pt]{article}

% sans serif font 
%\renewcommand{\familydefault}{\sfdefault}


%title page stuff
\newcommand{\HRule}{\rule{\linewidth}{0.5mm}}

% margins
\usepackage[top=2.5cm, bottom=2.5cm, left=4cm, right=2.5cm]{geometry}
\geometry{a4paper} % or letterpaper (US) or a5paper or....
%\geometry{margin=1in}

\usepackage{fancyhdr}
\pagestyle{fancyplain}
\fancyfoot[C]{\thepage}                 % page number in centre of foot.
\fancyfoot[R]{\small{M. J. Hawes, 10238908}}	% right of footer. 
\rhead{}

% \usepackage{setspace}           
% \onehalfspacing               % Sets the line spacing to 1.5

% ------------------------------------------------------------------------------
% provides comment functionality
\usepackage{verbatim}

% ------------------------------------------------------------------------------
% setting up harvard style refs
%\usepackage{harvard}
%\bibliographystyle{harvard}
\bibliographystyle{plain}

\usepackage{cite}

% ------------------------------------------------------------------------------
% needed for code listings
\usepackage{color}
\usepackage{xcolor}
\usepackage{listings}

\usepackage{caption}
\DeclareCaptionFont{black}{\color{black}}
\DeclareCaptionFormat{listing}{\colorbox{white}{\parbox{\textwidth}{#1#2#3}}}
\captionsetup[lstlisting]{format=listing,labelfont=black,textfont=black}

\lstset{ %
language=C,                % choose the language of the code
basicstyle=\footnotesize,       % the size of the fonts that are used for the code
numbers=left,                   % where to put the line-numbers
numberstyle=\footnotesize,      % the size of the fonts that are used for the line-numbers
stepnumber=1,                   % the step between two line-numbers. If it is 1 each line will be numbered
numbersep=5pt,                  % how far the line-numbers are from the code
backgroundcolor=\color{black!6!white},  % choose the background color. You must add \usepackage{color}
showspaces=false,               % show spaces adding particular underscores
showstringspaces=false,         % underline spaces within strings
showtabs=false,                 % show tabs within strings adding particular underscores
frame=single,           % adds a frame around the code
tabsize=2,          % sets default tabsize to 2 spaces
captionpos=b,           % sets the caption-position to bottom
breaklines=true,        % sets automatic line breaking
breakatwhitespace=true,    % sets if automatic breaks should only happen at whitespace
%escapeinside={\%*}{*)}          % if you want to add a comment within your code
}

% ------------------------------------------------------------------------------
% Graphics stuff for images
\usepackage{graphicx}
\DeclareGraphicsExtensions{.jpg,.jpeg,.png}
\graphicspath{ {./imgs/} }

% for graphs
\usepackage{tikz}
\usepackage{pgfplots}

\usepackage{epstopdf}		% .eps to .pdf

\usepackage{float}
%\floatstyle{boxed}
\restylefloat{figure}
\usepackage{caption}
\usepackage{subcaption}

\renewcommand{\thesubfigure}{\roman{subfigure}}
\captionsetup[subfigure]{labelformat=simple, labelsep=colon}

% Caption stuff
\begin{comment}
\DeclareCaptionFormat{myformat}{%
  \colorbox{lightgray!30}{\parbox{\dimexpr\textwidth-2\fboxsep-2\fboxrule\relax}{#1#2#3}}
} 
\captionsetup[figure]{format=myformat}
\captionsetup[lstlisting]{format=myformat}
\captionsetup[table]{format=myformat}
\end{comment}

% ------------------------------------------------------------------------------
% Maths packages
\usepackage{amsmath}

% ------------------------------------------------------------------------------
% Glossary packages
\newcommand*{\glossfirstformat}[1]{\textbf{#1}}

\usepackage[]{glossaries}
\renewcommand*{\glspostdescription}{}
\glossarystyle{list}
\makeglossaries

\renewcommand{\glsdisplayfirst}[4]{\glossfirstformat{#1#4}}
\renewcommand*{\glsnamefont}{\sffamily}
\renewcommand*{\glossarymark}[1]{}

\newglossaryentry{Makefile}
{
    name={Makefile},
    description={},
    plural={Makefiles}
}

\newglossaryentry{thread}
{
    name={thread},
    description={},
    plural={threads}
}

\newglossaryentry{multi-threaded algorithm}
{
    name={multi-threaded algorithm},
    description={},
    plural={multi-threaded algorithms}
}

\newglossaryentry{scheduling}
{
    name={scheduling},
    description={},    
}

\newglossaryentry{worker-thread}
{
    name={worker-thread},
    description={},
    plural={worker-threads}
}

\newglossaryentry{monitor-thread}
{
    name={monitor-thread},
    description={},
    plural={monitor-threads}
}

\newglossaryentry{mutex}
{
    name={mutex},
    description={},
    plural={mutexes}
}

\newglossaryentry{barrier}
{
    name={barrier},
    description={},
    plural={barriers}
}

\newglossaryentry{conditional-variable}
{
    name={conditional-variable},
    description={},
    plural={conditional-variables}
}
\newglossaryentry{load-balancing}
{
    name={load-balancing},
    description={},
}

\newglossaryentry{work-stealing}
{
    name={work-stealing},
    description={A processor which is starved of work attempts to ``steal'' 
                 work from other processors},
}

\newglossaryentry{thief}
{
    name={thief},
    description={},
    plural={thieves}
}

\newglossaryentry{victim}
{
    name={victim},
    description={},
    plural={victims}
}

\newglossaryentry{ready-deque}
{
    name={ready-deque},
    description={},
    plural={ready-deques}
}

\newglossaryentry{non-blocking}
{
    name={non-blocking},
    description={},
}

\newglossaryentry{steal-operation}
{
    name={steal-operation},
    description={},
}

\newglossaryentry{work-sharing}
{
    name={work-stealing},
    description={A processor which creates new work attempts to migrate it to 
                 another underutilised processor at creation time},
}

\newglossaryentry{locality}
{
    name={locality},
    description={},
}

\newglossaryentry{julia-set}
{
    name={julia-set},
    description={},
    plural={julia-sets}
}

\newglossaryentry{filled julia-set}
{
    name={filled julia-set},
    description={},
    plural={filled julia-sets}
}

\newglossaryentry{critical-orbit}
{
    name={critical-orbit},
    description={},
    plural={critical-orbits}
}

\newglossaryentry{fractal}
{
    name={fractal},
    description={},
    plural={fractals}
}

\newglossaryentry{complex-number}
{
    name={complex-numer},
    description={},
    plural={complex-numbers}
}

\newglossaryentry{real-world-fractal}
{
    name={real-world-fractal},
    description={},
    plural={real-world-fractals}
}

\newglossaryentry{mathematical-fractal}
{
    name={mathematical-fractal},
    description={},
    plural={mathematical-fractals}
}

\newglossaryentry{scale-invariance}
{
    name={scale-invariance},
    description={},
}

\newglossaryentry{self-similarity}
{
    name={self-similarity},
    description={},
}

\newglossaryentry{exact self-similarity}
{
    name={exact self-similarity},
    description={},
}

\newglossaryentry{quasi self-similarity}
{
    name={quasi self-similarity},
    description={},
    plural={quasi self-similar}
}



% ------------------------------------------------------------------------------
% Appendix packages
\usepackage[toc,page]{appendix}




% ------------------------------------------------------------------------------
%title
\title{
\huge A Work-Stealing Scheduling Technique Applied to Computing the Mandelbrot Set.
}

\author{
  Martin J Hawes\\
  The University of Hertfordshire \\
}

% ------------------------------------------------------------------------------
% - Document starts here
% ------------------------------------------------------------------------------

\begin{document}
\maketitle
\date{}

% ---------------------------------------------------------------------------------------------------------------------------------------------------------------
\section{Abstract}
This is the abstract. Lets cite something \cite[p.~201]{blank}.

This is another citation which should look a bit nicer \cite{jmcguin}.

%\begin{figure}[p]
%\centering
%\includegraphics[scale=0.5]{test}
%\caption{Test Image}
%\label{fig1}
%\end{figure}

hello reference of fig \ref{fig1}

\nocite{another}

% ---------------------------------------------------------------------------------------------------------------------------------------------------------------
\section{Acknowledgements}

% ---------------------------------------------------------------------------------------------------------------------------------------------------------------
\tableofcontents

% ---------------------------------------------------------------------------------------------------------------------------------------------------------------
% ---------------------------------------------------------------------------------------------------------------------------------------------------------------
% ---------------------------------------------------------------------------------------------------------------------------------------------------------------
\chapter{Introduction}

% ---------------------------------------------------------------------------------------------------------------------------------------------------------------
% ---------------------------------------------------------------------------------------------------------------------------------------------------------------
% ---------------------------------------------------------------------------------------------------------------------------------------------------------------
\chapter{Background Research}
% ---------------------------------------------------------------------------------------------------------------------------------------------------------------
\section{Scheduling Techniques for Multi-Threaded Computations}
This section briefly describes the problems associated with scheduling Multi-Threaded Algorithms and 
the major paradigms that have surfaced.

% explain problems here...... is this better off in the intro??
In order to efficiently utilize a parallel computer architecture it is desirable to minimize
the amount of time a processor spends idle or performing other logistical tasks, I.e not doing work. 
When a computation's \textit{concurrent sub-tasks} or \textit{threads} incur a regular cost in processor
time, each processor can simply have the same amount of work assigned to them. When the computation has
more irregular or dynamically growing sub-tasks a problem arises that results in 
processors becoming idle while others still remain working. This problem is refereed to as
\textit{load balancing} and can be described as a form of dynamic scheduling that ensures each processor 
spends approximately the same amount of time working. However, this introduces a new overhead in that 
processors now have to deal with these scheduling tasks as well. 

When considering the scheduling of multi-threaded computation, two major techniques have been used.
These are \textit{work-sharing} and \textit{work-stealing}.
When scheduling using a work-sharing approach, a thread that creates new threads
attempts to migrate these to another underutilised processor. In work-stealing a
thread that is starved of work, attempts to ``steal'' work from other processors. Both 
techniques intend to promote balanced work-load across all processors, however in Work-Stealing
the frequency of thread migrations is lower\cite[p.~2]{blumleis}. When all processors have a 
high work-load and no need to "steal" this is extremely useful because threads need not get 
migrated at all. With work-sharing thread migration occurs each time a thread spawns \cite[p.~2]{blumleis}.



%This is especially the case
%when all processors have a high work-load and no reason to "steal" from others. 

% ---------------------------------------------------------------------------------------------------------------------------------------------------------------
\section{The Work-Stealing Technique - Described in Depth}

% ---------------------------------------------------------------------------------------------------------------------------------------------------------------
\section{The Mandelbrot Set}

% ---------------------------------------------------------------------------------------------------------------------------------------------------------------
\section{Programming Tools}

% ---------------------------------------------------------------------------------------------------------------------------------------------------------------
% ---------------------------------------------------------------------------------------------------------------------------------------------------------------
% ---------------------------------------------------------------------------------------------------------------------------------------------------------------
\chapter{Main Chapters}
% ---------------------------------------------------------------------------------------------------------------------------------------------------------------
\section{Design of the Algorithm}

% ---------------------------------------------------------------------------------------------------------------------------------------------------------------
\section{The Implementation}

% ---------------------------------------------------------------------------------------------------------------------------------------------------------------
\subsection{An Algorithm to Compute the Mandelbrot Set}

% ---------------------------------------------------------------------------------------------------------------------------------------------------------------
\subsection{A Concurrent Algorithm to Compute the Mandelbrot Set}

% ---------------------------------------------------------------------------------------------------------------------------------------------------------------
\section{Features for Demonstration}

% ---------------------------------------------------------------------------------------------------------------------------------------------------------------
\section{Validation and Verification}

% ---------------------------------------------------------------------------------------------------------------------------------------------------------------
% ---------------------------------------------------------------------------------------------------------------------------------------------------------------
% ---------------------------------------------------------------------------------------------------------------------------------------------------------------
\chapter{Discussion and Evaluation}
% ---------------------------------------------------------------------------------------------------------------------------------------------------------------
\section{Analysis of the Algorithm}

% ---------------------------------------------------------------------------------------------------------------------------------------------------------------
\section{Reflection on Project}

% ---------------------------------------------------------------------------------------------------------------------------------------------------------------
% ---------------------------------------------------------------------------------------------------------------------------------------------------------------
% ---------------------------------------------------------------------------------------------------------------------------------------------------------------
\chapter{Resources}

% refs\bibs:
\bibliographystyle{harvard}
\bibliography{refs}

% ---------------------------------------------------------------------------------------------------------------------------------------------------------------
% ---------------------------------------------------------------------------------------------------------------------------------------------------------------
% ---------------------------------------------------------------------------------------------------------------------------------------------------------------
\chapter{Appendices}

\end{document}
