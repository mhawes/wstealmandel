% ------------------------------------------------------------------------------
% includes preamble stuff
\documentclass[a4paper,11pt]{report}
%\documentclass[a4paper,11pt]{article}

% sans serif font 
%\renewcommand{\familydefault}{\sfdefault}


%title page stuff
\newcommand{\HRule}{\rule{\linewidth}{0.5mm}}

% margins
\usepackage[top=2.5cm, bottom=2.5cm, left=4cm, right=2.5cm]{geometry}
\geometry{a4paper} % or letterpaper (US) or a5paper or....
%\geometry{margin=1in}

\usepackage{fancyhdr}
\pagestyle{fancyplain}
\fancyfoot[C]{\thepage}                 % page number in centre of foot.
\fancyfoot[R]{\small{M. J. Hawes, 10238908}}	% right of footer. 
\rhead{}

% \usepackage{setspace}           
% \onehalfspacing               % Sets the line spacing to 1.5

% ------------------------------------------------------------------------------
% provides comment functionality
\usepackage{verbatim}

% ------------------------------------------------------------------------------
% setting up harvard style refs
%\usepackage{harvard}
%\bibliographystyle{harvard}
\bibliographystyle{plain}

\usepackage{cite}

% ------------------------------------------------------------------------------
% needed for code listings
\usepackage{color}
\usepackage{xcolor}
\usepackage{listings}

\usepackage{caption}
\DeclareCaptionFont{black}{\color{black}}
\DeclareCaptionFormat{listing}{\colorbox{white}{\parbox{\textwidth}{#1#2#3}}}
\captionsetup[lstlisting]{format=listing,labelfont=black,textfont=black}

\lstset{ %
language=C,                % choose the language of the code
basicstyle=\footnotesize,       % the size of the fonts that are used for the code
numbers=left,                   % where to put the line-numbers
numberstyle=\footnotesize,      % the size of the fonts that are used for the line-numbers
stepnumber=1,                   % the step between two line-numbers. If it is 1 each line will be numbered
numbersep=5pt,                  % how far the line-numbers are from the code
backgroundcolor=\color{black!6!white},  % choose the background color. You must add \usepackage{color}
showspaces=false,               % show spaces adding particular underscores
showstringspaces=false,         % underline spaces within strings
showtabs=false,                 % show tabs within strings adding particular underscores
frame=single,           % adds a frame around the code
tabsize=2,          % sets default tabsize to 2 spaces
captionpos=b,           % sets the caption-position to bottom
breaklines=true,        % sets automatic line breaking
breakatwhitespace=true,    % sets if automatic breaks should only happen at whitespace
%escapeinside={\%*}{*)}          % if you want to add a comment within your code
}

% ------------------------------------------------------------------------------
% Graphics stuff for images
\usepackage{graphicx}
\DeclareGraphicsExtensions{.jpg,.jpeg,.png}
\graphicspath{ {./imgs/} }

% for graphs
\usepackage{tikz}
\usepackage{pgfplots}

\usepackage{epstopdf}		% .eps to .pdf

\usepackage{float}
%\floatstyle{boxed}
\restylefloat{figure}
\usepackage{caption}
\usepackage{subcaption}

\renewcommand{\thesubfigure}{\roman{subfigure}}
\captionsetup[subfigure]{labelformat=simple, labelsep=colon}

% Caption stuff
\begin{comment}
\DeclareCaptionFormat{myformat}{%
  \colorbox{lightgray!30}{\parbox{\dimexpr\textwidth-2\fboxsep-2\fboxrule\relax}{#1#2#3}}
} 
\captionsetup[figure]{format=myformat}
\captionsetup[lstlisting]{format=myformat}
\captionsetup[table]{format=myformat}
\end{comment}

% ------------------------------------------------------------------------------
% Maths packages
\usepackage{amsmath}

% ------------------------------------------------------------------------------
% Glossary packages
\newcommand*{\glossfirstformat}[1]{\textbf{#1}}

\usepackage[]{glossaries}
\renewcommand*{\glspostdescription}{}
\glossarystyle{list}
\makeglossaries

\renewcommand{\glsdisplayfirst}[4]{\glossfirstformat{#1#4}}
\renewcommand*{\glsnamefont}{\sffamily}
\renewcommand*{\glossarymark}[1]{}

\newglossaryentry{Makefile}
{
    name={Makefile},
    description={},
    plural={Makefiles}
}

\newglossaryentry{thread}
{
    name={thread},
    description={},
    plural={threads}
}

\newglossaryentry{multi-threaded algorithm}
{
    name={multi-threaded algorithm},
    description={},
    plural={multi-threaded algorithms}
}

\newglossaryentry{scheduling}
{
    name={scheduling},
    description={},    
}

\newglossaryentry{worker-thread}
{
    name={worker-thread},
    description={},
    plural={worker-threads}
}

\newglossaryentry{monitor-thread}
{
    name={monitor-thread},
    description={},
    plural={monitor-threads}
}

\newglossaryentry{mutex}
{
    name={mutex},
    description={},
    plural={mutexes}
}

\newglossaryentry{barrier}
{
    name={barrier},
    description={},
    plural={barriers}
}

\newglossaryentry{conditional-variable}
{
    name={conditional-variable},
    description={},
    plural={conditional-variables}
}
\newglossaryentry{load-balancing}
{
    name={load-balancing},
    description={},
}

\newglossaryentry{work-stealing}
{
    name={work-stealing},
    description={A processor which is starved of work attempts to ``steal'' 
                 work from other processors},
}

\newglossaryentry{thief}
{
    name={thief},
    description={},
    plural={thieves}
}

\newglossaryentry{victim}
{
    name={victim},
    description={},
    plural={victims}
}

\newglossaryentry{ready-deque}
{
    name={ready-deque},
    description={},
    plural={ready-deques}
}

\newglossaryentry{non-blocking}
{
    name={non-blocking},
    description={},
}

\newglossaryentry{steal-operation}
{
    name={steal-operation},
    description={},
}

\newglossaryentry{work-sharing}
{
    name={work-stealing},
    description={A processor which creates new work attempts to migrate it to 
                 another underutilised processor at creation time},
}

\newglossaryentry{locality}
{
    name={locality},
    description={},
}

\newglossaryentry{julia-set}
{
    name={julia-set},
    description={},
    plural={julia-sets}
}

\newglossaryentry{filled julia-set}
{
    name={filled julia-set},
    description={},
    plural={filled julia-sets}
}

\newglossaryentry{critical-orbit}
{
    name={critical-orbit},
    description={},
    plural={critical-orbits}
}

\newglossaryentry{fractal}
{
    name={fractal},
    description={},
    plural={fractals}
}

\newglossaryentry{complex-number}
{
    name={complex-numer},
    description={},
    plural={complex-numbers}
}

\newglossaryentry{real-world-fractal}
{
    name={real-world-fractal},
    description={},
    plural={real-world-fractals}
}

\newglossaryentry{mathematical-fractal}
{
    name={mathematical-fractal},
    description={},
    plural={mathematical-fractals}
}

\newglossaryentry{scale-invariance}
{
    name={scale-invariance},
    description={},
}

\newglossaryentry{self-similarity}
{
    name={self-similarity},
    description={},
}

\newglossaryentry{exact self-similarity}
{
    name={exact self-similarity},
    description={},
}

\newglossaryentry{quasi self-similarity}
{
    name={quasi self-similarity},
    description={},
    plural={quasi self-similar}
}



% ------------------------------------------------------------------------------
% Appendix packages
\usepackage[toc,page]{appendix}




% ------------------------------------------------------------------------------
%title
\title{
\huge A Work-Stealing Scheduling Technique Applied to Computing the Mandelbrot Set.
}

\author{
  Martin J Hawes\\
  \\
  Department of Computer Science, \\
  The University of Hertfordshire \\ 
  \\
  \textit{hawesmartin@googlemail.com}\\
}

% ------------------------------------------------------------------------------
% - Document starts here
% ------------------------------------------------------------------------------

\begin{document}
\maketitle
\date{}

% ---------------------------------------------------------------------------------------------------------------------------------------------------------------
\pagenumbering{Roman}
\section*{Abstract}


% ---------------------------------------------------------------------------------------------------------------------------------------------------------------
\section*{Acknowledgements}


% ---------------------------------------------------------------------------------------------------------------------------------------------------------------
\clearpage
\pagenumbering{arabic}
\setcounter{page}{1}
\tableofcontents

% ---------------------------------------------------------------------------------------------------------------------------------------------------------------
% ---------------------------------------------------------------------------------------------------------------------------------------------------------------
% ---------------------------------------------------------------------------------------------------------------------------------------------------------------
\chapter{Introduction}

% Identify subject area and clearly state the arena of interest this report investigates.
    % * state of the field. 
    % * Talk about highly parallel computing and its applications. Be broad.

% Identify some of the key pieces of literature or applications for the subject area I have chosen.

% Talk about the specific investigated area's and what the report specifically talks about.
    % * i.e. which work-stealing schemes are implemented, which Mandelbrot algorithm is identified.

% Rationale for doing above said things and aims of project.

% Disclaimer section i.e. This report assumes prior knowledge of blah blah blah and is written
%   with whomever in mind and so on.

% ---------------------------------------------------------------------------------------------------------------------------------------------------------------
% ---------------------------------------------------------------------------------------------------------------------------------------------------------------
% ---------------------------------------------------------------------------------------------------------------------------------------------------------------
\chapter{Problem Background}
% ---------------------------------------------------------------------------------------------------------------------------------------------------------------

\section{Run-Time Scheduling Techniques for Multi-Threaded Computations}
This section briefly describes the problems associated with scheduling Multi-Threaded Algorithms at run-time and
the major paradigms that have surfaced.

% explain problems here...... is this better off in the intro??
To efficiently utilise a parallel computer architecture it is desirable to minimise
the amount of time a processor spends idle or performing other logistical tasks, i.e not doing work. 
When a computation's \textit{concurrent sub-tasks} or \textit{threads} incur a regular cost in processor
time, each processor can simply have the same amount of work assigned to them. When the computation has
more irregular or dynamically growing sub-tasks a problem arises resulting in 
processors becoming idle while others still remain working. The solution to this problem is referred to as
\textit{load balancing} and can be described as a form of dynamic scheduling that ensures each processor 
spends approximately the same amount of time working. This means processors generally spend
less time idle, however have to deal with scheduling overheads as a trade-off.

When considering the scheduling of multi-threaded computations, two major load balancing techniques have been used.
These are \textit{work-sharing} and \textit{work-stealing}.

\begin{itemize}
\item \textbf{Work-Sharing:} A processor which creates new work attempts to migrate it to another underutilised processor at creation time. 
\item \textbf{Work-Stealing:} A processor which is starved of work attempts to ``steal'' work from other processors. 
\end{itemize}

Both techniques intend to promote balanced work-load across all processors, however in Work-Stealing
the frequency of work migrations is lower. When all processors have a 
high work-load and no need to ``steal" this becomes useful because threads need not get 
migrated at all. With work-sharing work migration occurs each time new work is created \cite{blumleis}.
This also suggests that work-stealing promotes better locality and grouping of sub-tasks, as spawned work 
stays with the same processor until stolen.

% ---------------------------------------------------------------------------------------------------------------------------------------------------------------
\section{The Mandelbrot Set}

%\subsection{Background}
The Mandelbrot set is a set of complex numbers which when plotted produce a spectacular and recognisable shape as illustrated in figure~\ref{fig:mandelimg}.
It is often presented as a colourful and striking image and has been described by some as the most beautiful object in all of mathematics \cite[p.~234]{chaosfract}.
The complex numbers that comprise the set are closely related to \textit{Julia Sets}. 
In-fact the Mandelbrot set can be described as a catalogue of Julia Sets which, when plotted, all points are \textit{connected}, 
forming a \textit{single, unbroken shape} \cite[p.~177]{fractimg}.

The set is named for the mathematician \textit{Benoit Mandelbrot}, who discovered it in 1980 \cite{fracnature , fractimg}. He was a pioneer in the study of 
fractal geometry and also coined the term \textit{fractal}, of which both the Mandelbrot set, and Julia sets are examples of. 

In this section I will give a more detailed explanation of the areas mentioned here. 

\begin{figure}[h]
  \caption{A rendering of The Mandelbrot Set generated using the program ``fraqtive"\cite{fraqtive}.}
  \label{fig:mandelimg}
  \centering
    \includegraphics[width=1\textwidth]{mandelbrot}
\end{figure}

\subsection*{Fractals and Self-Similarity} 
%TODO: this section needs citations all over the shop!
A fractal is a means of describing shapes which are more complex than classical geometric shapes. The leaves of a pine tree or the forks of a 
lightning bolt are obvious examples of real things that fractals allow us to more faithfully describe. 
These \textit{real world fractals} are similar to \textit{mathematical fractals}, 
of which the Mandelbrot set is an example, but differ in that they do not display the property of \textit{Scale Invariance}. 

Fractals have a fractional dimension. Unlike shapes with topological dimension, for instance a square has two dimensions, 
a fractal's dimension is of a non integer value.

A property of fractals (but not all) is \textit{Self-Similarity}, where the shape is comprised of smaller ``copies" of itself. 
This is known as \textit{Exact Self-Similarity} and means that the shape is identical at any scale.
A well-known example of this is the \textit{Triadic Koch Snowflake} which is a fractal constructed using equilateral triangles. 
It is important to note here that The Mandelbrot set does not quite show the same property, it is said to be 
\textit{Quasi Self-Similar}. This means that the shape is approximately similar at all scales, in that the shape is replicated but in a slightly distorted
form with each ``copy".

It turns out there are many rather useful applications for fractals. To name a few; computer game graphics, %FIND MORE EXAMPLES WITH CITATIONS.

\subsection*{Julia Sets}

To understand the basis of the Mandelbrot set it is first necessary to understand it's relation to \textit{Julia Sets}.
The function in equation~\ref{eq:julia1} is iterated infinitely where \(c\) is fixed.
The \textit{filled Julia Set} is comprised of all values of \(z_0\) where the result is bounded and does not tend towards infinity.
The \textit{Julia Set} is comprised of those members of the \textit{filled Julia Set} which lie on the boundary \cite{chaosfract}.
In the interest of keeping this report readable, and because filled Julia Sets are more relevant, filled Julia sets will be 
referred to simply as \textit{Julia Sets}.

% express the set notation for f(x) -> infinity.???? TODO
\begin{equation}\label{eq:julia1}
f(z) = z^2 + c
\end{equation}

With regards to the Mandelbrot Set I am interested in Julia Sets in which the values \(c\) and \(z\) used are expressed as a complex number. 
Figure~\ref{fig:juliaimgs} illustrates some examples of such sets. 

\begin{figure}[h]
\centering
\begin{subfigure}[b]{0.48\textwidth}
  \centering    
  \includegraphics[width=\textwidth]{julia-con}
  \caption{
    \tiny The Julia Set where \(c = -0.1 + 0.649i\)
  }
  \label{fig:juliaimgcon}
\end{subfigure}
~ %spacer
\begin{subfigure}[b]{0.48\textwidth}
  \centering
  \includegraphics[width=\textwidth]{julia-ncon}
  \caption{
    \tiny The Julia Set where \(c = -0.75 + 0.03i\)
  }
  \label{fig:juliaimgncon}
\end{subfigure}
% full caption
\caption{
  Two Julia Sets rendered using the program ``fraqtive"\cite{fraqtive}. 
  Figure~\ref{fig:juliaimgcon} is a member of the Mandelbrot set, 
  figure~\ref{fig:juliaimgncon} is not.
}
\label{fig:juliaimgs}
\end{figure}

\subsection*{Computing the Mandelbrot Set}

The set is comprised of those Julia Sets which are \textit{connected}. In order to determine whether a Julia Set possesses this property,
we need only compute the result for \(z_0\). If this tends towards infinity the value \(c\) is not a member of the Mandelbrot Set. If the result
is bounded, then it is a member. This is known as the \textit{critical orbit} and is useful because it means we do not have to compute
the entire Julia Set for each value of \(c\).

So the Mandelbrot set can be computed by iterating all possible values of \(c\) for the function in equation~\ref{eq:julia1} where \(z\) is the \textit{critical
orbit}. Because the set of all possible values of \(c\) is infinite, and computers have a finite amount of resources, this set needs to be approximated. 
This can be done using a raster plane which takes samples of the complex plane at regular intervals. 

% TALK ABOUT VARIOUS ALGORITHMS PRESENTED TODO
Level set method \cite[p.~188]{fractimg} Continuous Potential Method \cite[p.~191]{fractimg}

% ---------------------------------------------------------------------------------------------------------------------------------------------------------------
\section{The Work-Stealing Technique - Described in Depth}

As described above, Work-Stealing is a load balancing technique which allows work starved processors to acquire scheduled work from other processors. 

Each processor has a number of assigned tasks to complete. In general, a processor acquires its work from here.
However, once these tasks are exhausted, the processor becomes a \textit{thief} and chooses a \textit{victim} to steal from. 
The method used to choose a victim is implementation specific, for instance some implementations adopt a random scheme \cite{blumleis , jliff, narora}.
If the processor successfully steals work it relinquishes its thief state and returns to doing work.
If the steal attempt is unsuccessful, for instance when the victim has no work or is blocked, the processor tries again
until it is determined that there is no work remaining in the entire network. 

Figure~\ref{fig:stealoperation} illustrates the result of a successful steal operation in which work-starved processor \textit{p1} transfers a piece
of work from \textit{p0's} work list to its own. %do more here.

\begin{figure}[h]
\centering
\begin{subfigure}[b]{0.4\textwidth}
  \centering    
%  \includegraphics[width=\textwidth]{stealreq}
  % Graphic for TeX using PGF
% Title: /media/martin/69f663af-aa17-4b22-836c-5646358695f1/fyp/wstealmandel/writeup/diagrams/stealreq.dia
% Creator: Dia v0.97.2
% CreationDate: Fri Feb 15 17:19:52 2013
% For: martin
% \usepackage{tikz}
% The following commands are not supported in PSTricks at present
% We define them conditionally, so when they are implemented,
% this pgf file will use them.
\ifx\du\undefined
  \newlength{\du}
\fi
\setlength{\du}{15\unitlength}
\begin{tikzpicture}
\pgftransformxscale{1.000000}
\pgftransformyscale{-1.000000}
\definecolor{dialinecolor}{rgb}{0.000000, 0.000000, 0.000000}
\pgfsetstrokecolor{dialinecolor}
\definecolor{dialinecolor}{rgb}{1.000000, 1.000000, 1.000000}
\pgfsetfillcolor{dialinecolor}
\pgfsetlinewidth{0.100000\du}
\pgfsetdash{}{0pt}
\pgfsetdash{}{0pt}
\pgfsetmiterjoin
\definecolor{dialinecolor}{rgb}{1.000000, 1.000000, 1.000000}
\pgfsetfillcolor{dialinecolor}
\fill (1.000000\du,11.800000\du)--(1.000000\du,13.800000\du)--(3.000000\du,13.800000\du)--(3.000000\du,11.800000\du)--cycle;
\definecolor{dialinecolor}{rgb}{0.000000, 0.000000, 0.000000}
\pgfsetstrokecolor{dialinecolor}
\draw (1.000000\du,11.800000\du)--(1.000000\du,13.800000\du)--(3.000000\du,13.800000\du)--(3.000000\du,11.800000\du)--cycle;
% setfont left to latex
\definecolor{dialinecolor}{rgb}{0.000000, 0.000000, 0.000000}
\pgfsetstrokecolor{dialinecolor}
\node[anchor=west] at (1.600000\du,13.000000\du){p0};
\pgfsetlinewidth{0.100000\du}
\pgfsetdash{}{0pt}
\pgfsetdash{}{0pt}
\pgfsetmiterjoin
\definecolor{dialinecolor}{rgb}{1.000000, 1.000000, 1.000000}
\pgfsetfillcolor{dialinecolor}
\fill (1.000000\du,7.000000\du)--(1.000000\du,10.800000\du)--(3.000000\du,10.800000\du)--(3.000000\du,7.000000\du)--cycle;
\definecolor{dialinecolor}{rgb}{0.000000, 0.000000, 0.000000}
\pgfsetstrokecolor{dialinecolor}
\draw (1.000000\du,7.000000\du)--(1.000000\du,10.800000\du)--(3.000000\du,10.800000\du)--(3.000000\du,7.000000\du)--cycle;
% setfont left to latex
\definecolor{dialinecolor}{rgb}{0.000000, 0.000000, 0.000000}
\pgfsetstrokecolor{dialinecolor}
\node[anchor=west] at (0.000000\du,10.400000\du){};
\pgfsetlinewidth{0.100000\du}
\pgfsetdash{}{0pt}
\pgfsetdash{}{0pt}
\pgfsetbuttcap
{
\definecolor{dialinecolor}{rgb}{0.000000, 0.000000, 0.000000}
\pgfsetfillcolor{dialinecolor}
% was here!!!
\definecolor{dialinecolor}{rgb}{0.000000, 0.000000, 0.000000}
\pgfsetstrokecolor{dialinecolor}
\draw (1.400000\du,8.400000\du)--(2.600000\du,8.400000\du);
}
\pgfsetlinewidth{0.100000\du}
\pgfsetdash{}{0pt}
\pgfsetdash{}{0pt}
\pgfsetbuttcap
{
\definecolor{dialinecolor}{rgb}{0.000000, 0.000000, 0.000000}
\pgfsetfillcolor{dialinecolor}
% was here!!!
\definecolor{dialinecolor}{rgb}{0.000000, 0.000000, 0.000000}
\pgfsetstrokecolor{dialinecolor}
\draw (1.400000\du,8.800000\du)--(2.600000\du,8.800000\du);
}
\pgfsetlinewidth{0.100000\du}
\pgfsetdash{}{0pt}
\pgfsetdash{}{0pt}
\pgfsetbuttcap
{
\definecolor{dialinecolor}{rgb}{0.000000, 0.000000, 0.000000}
\pgfsetfillcolor{dialinecolor}
% was here!!!
\definecolor{dialinecolor}{rgb}{0.000000, 0.000000, 0.000000}
\pgfsetstrokecolor{dialinecolor}
\draw (1.400000\du,9.200000\du)--(2.600000\du,9.200000\du);
}
\pgfsetlinewidth{0.100000\du}
\pgfsetdash{}{0pt}
\pgfsetdash{}{0pt}
\pgfsetbuttcap
{
\definecolor{dialinecolor}{rgb}{0.000000, 0.000000, 0.000000}
\pgfsetfillcolor{dialinecolor}
% was here!!!
\definecolor{dialinecolor}{rgb}{0.000000, 0.000000, 0.000000}
\pgfsetstrokecolor{dialinecolor}
\draw (1.400000\du,9.600000\du)--(2.600000\du,9.600000\du);
}
\pgfsetlinewidth{0.100000\du}
\pgfsetdash{}{0pt}
\pgfsetdash{}{0pt}
\pgfsetbuttcap
{
\definecolor{dialinecolor}{rgb}{0.000000, 0.000000, 0.000000}
\pgfsetfillcolor{dialinecolor}
% was here!!!
\definecolor{dialinecolor}{rgb}{0.000000, 0.000000, 0.000000}
\pgfsetstrokecolor{dialinecolor}
\draw (1.400000\du,10.000000\du)--(2.600000\du,10.000000\du);
}
\pgfsetlinewidth{0.100000\du}
\pgfsetdash{}{0pt}
\pgfsetdash{}{0pt}
\pgfsetbuttcap
{
\definecolor{dialinecolor}{rgb}{0.000000, 0.000000, 0.000000}
\pgfsetfillcolor{dialinecolor}
% was here!!!
\definecolor{dialinecolor}{rgb}{0.000000, 0.000000, 0.000000}
\pgfsetstrokecolor{dialinecolor}
\draw (1.400000\du,10.400000\du)--(2.600000\du,10.400000\du);
}
\pgfsetlinewidth{0.100000\du}
\pgfsetdash{}{0pt}
\pgfsetdash{}{0pt}
\pgfsetbuttcap
{
\definecolor{dialinecolor}{rgb}{0.000000, 0.000000, 0.000000}
\pgfsetfillcolor{dialinecolor}
% was here!!!
\pgfsetarrowsend{stealth}
\definecolor{dialinecolor}{rgb}{0.000000, 0.000000, 0.000000}
\pgfsetstrokecolor{dialinecolor}
\draw (2.000000\du,10.800000\du)--(2.000000\du,11.800000\du);
}
\pgfsetlinewidth{0.100000\du}
\pgfsetdash{}{0pt}
\pgfsetdash{}{0pt}
\pgfsetmiterjoin
\definecolor{dialinecolor}{rgb}{1.000000, 1.000000, 1.000000}
\pgfsetfillcolor{dialinecolor}
\fill (6.000000\du,11.800000\du)--(6.000000\du,13.800000\du)--(8.000000\du,13.800000\du)--(8.000000\du,11.800000\du)--cycle;
\definecolor{dialinecolor}{rgb}{0.000000, 0.000000, 0.000000}
\pgfsetstrokecolor{dialinecolor}
\draw (6.000000\du,11.800000\du)--(6.000000\du,13.800000\du)--(8.000000\du,13.800000\du)--(8.000000\du,11.800000\du)--cycle;
% setfont left to latex
\definecolor{dialinecolor}{rgb}{0.000000, 0.000000, 0.000000}
\pgfsetstrokecolor{dialinecolor}
\node[anchor=west] at (6.600000\du,13.000000\du){p1};
\pgfsetlinewidth{0.100000\du}
\pgfsetdash{}{0pt}
\pgfsetdash{}{0pt}
\pgfsetmiterjoin
\definecolor{dialinecolor}{rgb}{1.000000, 1.000000, 1.000000}
\pgfsetfillcolor{dialinecolor}
\fill (6.000000\du,7.000000\du)--(6.000000\du,10.800000\du)--(8.000000\du,10.800000\du)--(8.000000\du,7.000000\du)--cycle;
\definecolor{dialinecolor}{rgb}{0.000000, 0.000000, 0.000000}
\pgfsetstrokecolor{dialinecolor}
\draw (6.000000\du,7.000000\du)--(6.000000\du,10.800000\du)--(8.000000\du,10.800000\du)--(8.000000\du,7.000000\du)--cycle;
% setfont left to latex
\definecolor{dialinecolor}{rgb}{0.000000, 0.000000, 0.000000}
\pgfsetstrokecolor{dialinecolor}
\node[anchor=west] at (5.000000\du,10.400000\du){};
\pgfsetlinewidth{0.100000\du}
\pgfsetdash{}{0pt}
\pgfsetdash{}{0pt}
\pgfsetbuttcap
{
\definecolor{dialinecolor}{rgb}{0.000000, 0.000000, 0.000000}
\pgfsetfillcolor{dialinecolor}
% was here!!!
\pgfsetarrowsend{stealth}
\definecolor{dialinecolor}{rgb}{0.000000, 0.000000, 0.000000}
\pgfsetstrokecolor{dialinecolor}
\draw (7.000000\du,10.800000\du)--(7.000000\du,11.800000\du);
}
\pgfsetlinewidth{0.100000\du}
\pgfsetdash{{\pgflinewidth}{0.200000\du}}{0cm}
\pgfsetdash{{\pgflinewidth}{0.200000\du}}{0cm}
\pgfsetmiterjoin
\pgfsetbuttcap
{
\definecolor{dialinecolor}{rgb}{0.000000, 0.000000, 0.000000}
\pgfsetfillcolor{dialinecolor}
% was here!!!
\pgfsetarrowsend{to}
{\pgfsetcornersarced{\pgfpoint{0.000000\du}{0.000000\du}}\definecolor{dialinecolor}{rgb}{0.000000, 0.000000, 0.000000}
\pgfsetstrokecolor{dialinecolor}
\draw (6.000000\du,12.800000\du)--(4.600000\du,12.800000\du)--(4.600000\du,8.400000\du)--(3.000000\du,8.400000\du);
}}
\end{tikzpicture}

  \caption{
    \tiny A steal request by the thief processor \textit{p1} on victim \textit{p0}.
  }
  \label{fig:stealreq}
\end{subfigure}
~~~~~~ %spacer
\begin{subfigure}[b]{0.4\textwidth}
  \centering
%  \includegraphics[width=\textwidth]{stealafter}
  \include{./imgs/stealafter}
  \caption{
    \tiny The attempt was successful and the work was re-assigned to \textit{p1}.
  }
  \label{fig:stealsuccess}
\end{subfigure}
\caption{
    A successful steal operation between a thief and its victim.
  }
\label{fig:stealoperation}
\end{figure}

Research has been conducted to explore its application in programming languages \cite{}, 
operating systems \cite{}, and high performance parallel computing \cite{}.

This section explores some schemes used to implement work-stealing in various settings. It is focused on overall design and 
techniques presented in related literature. 

\subsection*{Blumofe and Leiserson - A Randomized Work-Stealing Algorithm}

%TODO get citation for CILK language.
This scheme is geared towards computation of dynamically growing, fully strict, multi-threaded computations and is applied to the CILK
programming language and its runtime system \cite{blumleis}. 

Each thread maintains a \textit{ready-deque}, a double-ended queue of work waiting to be processed. 
Accesses to this queue are made either at the top for a steal operation, and at the bottom for a push operation
or when the next piece of work is required.
A thread becomes a \textbf{thief} when its ready-deque is empty, and randomly selects a \textbf{victim}; a thread
to attempt to steal work from. If this steal operation is successful it pushes the stolen work 
onto the bottom of its ready-deque and becomes a worker again. If not it tries, at random, to find another victim.

Because of the setting this scheme is designed for the algorithm needs to consider that
a piece of work can spawn children dynamically, which it may depend on completing to continue. 
When computing the Mandelbrot set in a concurrent environment, no such consideration is required 
as each point can be independently processed.

The ready-deque can be implemented in such a way that a thread need not be stopped in order for a steal operation 
to occur. This property is known as \textit{non-blocking} and only requires that the top end of the deque has atomic access,
while the bottom can freely be accessed by the thread which owns the deque \cite{narora}. This is useful because it reduces the overheads 
of a steal operation in that a working thread generally does not get interrupted. 
Further still, a non-blocking deque can be made efficient through use of a circular array \cite{circdeque}.

\subsection*{McGuiness - Render-Thread Algorithm}

This scheme is presented as part of McGuiness' Masters Thesis \cite{jmcguin} and is suited for parallel computation where each unit of work is independent 
from any other. Computation of the Mandelbrot set is given as an application of the algorithm. 
It is designed with cellular architectures in mind but can be applied to a shared memory architecture. 

The algorithm uses a set of \textit{worker-threads} (referred to by McGuiness as \textit{render-threads}), 
and a single \textit{monitor-thread}, to control the distribution of work. Each worker-thread is
initially given an equal share of the overall work-load before starting.

Each worker-thread maintains an estimated completion time for its assigned work-load. This is initially set to the maximum possible value and
is iteratively refined by calculating the average time taken to complete a piece of work. This metric is used as a policy for deciding which
thread is the most suitable candidate for the victim of a work-stealing operation.

When a worker-thread completes its assigned work a \textit{work-completed} signal is generated, its estimated
completion time is set to \textit{0}, and the thread is stopped. This thread will be referred to as the \textbf{thief}.
When the monitor thread detects this signal, it searches for the worker-thread with the longest estimated completion time, 
which will be referred to as the \textbf{victim}. The monitor-thread waits for the victim to complete its current piece of work before stopping it.
Its workload is then halved, having the other half re-asigned to the thief. Both the victim and the thief are restarted and continue doing work.
The monitor-thread returns to waiting for another work-completed signal and the process is repeated until no work remains.

% TODO talk about the fact that work need not be queued because all that is needed is a start line and a stop line.

% ---------------------------------------------------------------------------------------------------------------------------------------------------------------
% ---------------------------------------------------------------------------------------------------------------------------------------------------------------
% ---------------------------------------------------------------------------------------------------------------------------------------------------------------
\chapter{Main Sections}
% ---------------------------------------------------------------------------------------------------------------------------------------------------------------
\section{Tools}

This section discusses some of the programming and general tools which are candidates
for use in the implementation and for the good of the project as a whole.
The tools discussed in this section are all viable options for a Linux platform.

\subsection*{Languages}

\begin{itemize}
\item \textbf{C:} 
\item \textbf{C++:}
\item \textbf{Java:}
\end{itemize}

\subsection*{Concurrent Programming Libraries}
In order to implement Work-Stealing, support for programming threads 
with a suitable level of control is required.

\begin{itemize}
\item \textbf{POSIX Threads (pthreads):} Provides low level manipulation of threads for the C programming language \cite{pthreadover}. 
              It is a library based on IEEE standard 1003.1. Thread programming is achieved through use of a set of functions and data
              structures provided, such as \textit{mutexes}, \textit{semaphores}, \textit{ }. %TODO
             
\item \textbf{Open Multiprocessing (OpenMP):} Provides abstract thread programming interfaces for C, C++, and fortran.
              In general OpenMP only allows coarse grained manipulation of threads through features such as parallel loops %TODO
              \cite{ompvspthr}

\item \textbf{Java Threads:}
\end{itemize}

\subsection*{Graphical Output}
For the purpose of demonstrating that the program correctly generates a raster plane
of the Mandelbrot set, a graphical representation of the plane should be output.

\begin{itemize}
\item \textbf{PPM Output File:} The simplest option is to output to a Portable Pixel Map (PPM) file. It is 
              text-file based and easy to implement but produces rather large files. 
              The process of outputting to a text file is inherently serial in nature,
              so with large image resolutions processing takes a long time.
              There is support for both grey-scale (Portable Grey-scale Map format) and colour images. 
              \cite{ppmspec}
              
\item \textbf{GNU Plot:} A graph plotting package available for multiple operating systems. 
              Supports screen display or file output of both 2d and 3d graphics \cite{gnuplot}.
              There are programming interfaces available for various languages such as C \cite{gnuplotcint}, C++ \cite{gnuplotcppint}, 
              and Java \cite{gnuplotjint}.
              
%\item \textbf{OpenGL}
\end{itemize}

\subsection*{Other Tools}

Listed here are the major utility tools which have been used to make this project of a better over-all 
quality. 

\begin{itemize}
\item \textbf{GNU Make:}
\item         \LaTeX:
\item \textbf{Git:}
\end{itemize}

% ---------------------------------------------------------------------------------------------------------------------------------------------------------------
\section{Design of the Algorithm}

\begin{lstlisting}[label = li:mandelalgo, caption = A sequential algorithm to compute the Mandelbrot Set presented in pseudo code.]
compute_mandelbrot()
    FOR y = 0 TO height - 1 DO
        c_im := im_min + y * (im_max - im_min)/(height - 1)
        FOR x = 0 TO width - 1 DO
            c_re := re_min + x * (re_max - re_min)/(width - 1)
            plane[x][y] := is_in_set(c_re,c_im,max_iterations)
        END FOR
    END FOR
END

is_in_set(c_re,c_im,max_iterations)
    z_re := c_re
    z_im := c_im

    FOR i = 0 TO max_iterations DO
        IF( z_re^2 + z_im^2 > 4) THEN
            RETURN false
        END IF
    END FOR
    RETURN true 
END

\end{lstlisting}

% ---------------------------------------------------------------------------------------------------------------------------------------------------------------
\section{The Implementation}

% ---------------------------------------------------------------------------------------------------------------------------------------------------------------
\subsection{An Algorithm to Compute the Mandelbrot Set}

% talk about the static keyword in C for the array.
% talk about inlining some functions
% talk about 

% ---------------------------------------------------------------------------------------------------------------------------------------------------------------
\subsection{A Concurrent Algorithm to Compute the Mandelbrot Set}

% ---------------------------------------------------------------------------------------------------------------------------------------------------------------
\section{Features for Demonstration}

% ---------------------------------------------------------------------------------------------------------------------------------------------------------------
\section{Validation and Verification}

% ---------------------------------------------------------------------------------------------------------------------------------------------------------------
% ---------------------------------------------------------------------------------------------------------------------------------------------------------------
% ---------------------------------------------------------------------------------------------------------------------------------------------------------------
\chapter{Discussion and Evaluation}
% ---------------------------------------------------------------------------------------------------------------------------------------------------------------
\section{Analysis of the Algorithm}

% ---------------------------------------------------------------------------------------------------------------------------------------------------------------
\section{Reflection on Project}
\subsection{Further Work}
% ---------------------------------------------------------------------------------------------------------------------------------------------------------------
% ---------------------------------------------------------------------------------------------------------------------------------------------------------------
% ---------------------------------------------------------------------------------------------------------------------------------------------------------------
\chapter{Resources}

\nocite{*}

% refs\bibs:
\bibliography{refs}

% ---------------------------------------------------------------------------------------------------------------------------------------------------------------
% ---------------------------------------------------------------------------------------------------------------------------------------------------------------
% ---------------------------------------------------------------------------------------------------------------------------------------------------------------
\chapter{Appendices}

\end{document}
